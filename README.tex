\hypertarget{rlwe-homomorphic-encryption-scheme}{%
\section{RLWE Homomorphic Encryption
Scheme}\label{rlwe-homomorphic-encryption-scheme}}

This document provides a mathematical explanation of the RLWE (Ring
Learning With Errors) homomorphic encryption scheme. The provided Python
code demonstrates key generation, encryption, and various operations on
encrypted data.

\hypertarget{rlwe-encryption-parameters}{%
\subsection{RLWE Encryption
Parameters}\label{rlwe-encryption-parameters}}

The RLWE encryption scheme operates in a polynomial ring modulo a
polynomial modulus ( poly\_mod ). The ciphertext modulus is denoted as (
q ), and the plaintext modulus is denoted as ( t ). The encryption
scheme uses a base ( T ) for polynomial switching and a modulus ( p )
for modulus switching.

\hypertarget{parameters}{%
\subsubsection{Parameters}\label{parameters}}

The following parameters are defined:

\begin{itemize}
\tightlist
\item
  ( n ): Polynomial modulus degree
\item
  ( q ): Ciphertext modulus
\item
  ( t ): Plaintext modulus
\item
  ( T ): Base for polynomial switching
\item
  ( p ): Modulus for modulus switching
\end{itemize}

\hypertarget{key-generation}{%
\subsection{Key Generation}\label{key-generation}}

The key generation process involves generating a public key ( (a, b) )
and a secret key ( s ). The public key is used for encryption, and the
secret key is used for decryption.

\hypertarget{generating-secret-and-public-keys}{%
\subsubsection{Generating Secret and Public
Keys}\label{generating-secret-and-public-keys}}

The secret key ( s ) is a binary polynomial. The public key is generated
as follows:

\begin{itemize}
\tightlist
\item
  ( s ): Binary secret key polynomial
\item
  ( a ): Uniformly random polynomial
\item
  ( e ): Error polynomial sampled from a normal distribution
\item
  ( b ): ( \(a \cdot s - e \mod q\) )
\end{itemize}

\hypertarget{encryption}{%
\subsection{Encryption}\label{encryption}}

To encrypt a plaintext message ( m ) using the public key ( (a, b) ),
the following steps are performed:

\[
\begin{align*}
m &\equiv m \mod t \\
\text{delta} &= \frac{q}{t} \\
e1, e2 &= \text{Error polynomials sampled from a normal distribution} \\
u &= \text{Binary polynomial} \\
\text{ct} &\equiv (ct_0, ct_1) \\
ct_0 &= a \cdot u + e1 + \delta \cdot m \mod q \\
ct_1 &= b \cdot u + e2 \mod q \\
\end{align*}
\]

\hypertarget{homomorphic-operations}{%
\subsection{Homomorphic Operations}\label{homomorphic-operations}}

The RLWE encryption scheme supports various homomorphic operations, such
as addition and multiplication, on encrypted data.

\hypertarget{addition-of-ciphertext-and-plaintext}{%
\subsubsection{Addition of Ciphertext and
Plaintext}\label{addition-of-ciphertext-and-plaintext}}

To add a ciphertext ( ct ) and a plaintext ( pt ), the following
operation is performed:

\[
\begin{align*}
m &\equiv \text{pt} \mod t \\
\text{delta} &= \frac{q}{t} \\
e1 &= \text{Error polynomial} \\
\text{scaled\_m} &= \delta \cdot m \mod q \\
\text{new\_ct} &\equiv (\text{new\_ct\_0}, \text{ct\_1}) \\
\text{new\_ct\_0} &= ct_0 + \text{scaled\_m} \mod q \\
\end{align*}
\]

\hypertarget{addition-of-ciphertexts}{%
\subsubsection{Addition of Ciphertexts}\label{addition-of-ciphertexts}}

To add two ciphertexts ( ct1 ) and ( ct2 ), the following operation is
performed:

\[
\begin{align*}
\text{new\_ct} &\equiv (\text{new\_ct\_0}, \text{new\_ct\_1}) \\
\text{new\_ct\_0} &= ct1_0 + ct2_0 \mod q \\
\text{new\_ct\_1} &= ct1_1 + ct2_1 \mod q \\
\end{align*}
\]

\hypertarget{multiplication-of-ciphertext-and-plaintext}{%
\subsubsection{Multiplication of Ciphertext and
Plaintext}\label{multiplication-of-ciphertext-and-plaintext}}

To multiply a ciphertext ( ct ) by a plaintext ( pt ), the following
operation is performed:

\[
\begin{align*}
\text{scaled\_m} &= \delta \cdot m \mod q \\
\text{new\_ct} &\equiv (\text{new\_ct\_0}, \text{new\_ct\_1}) \\
\text{new\_ct\_0} &= ct_0 \cdot \text{scaled\_m} \mod q \\
\text{new\_ct\_1} &= ct_1 \cdot \text{scaled\_m} \mod q \\
\end{align*}
\]

\hypertarget{multiplication-of-ciphertexts}{%
\subsubsection{Multiplication of
Ciphertexts}\label{multiplication-of-ciphertexts}}

The multiplication of ciphertexts involves more complex steps and
relinearization keys. The provided Python code demonstrates two versions
of multiplication: version 1 and version 2.

\hypertarget{decryption}{%
\subsection{Decryption}\label{decryption}}

To decrypt a ciphertext and obtain the original plaintext message, the
following steps are performed:

\[
\begin{align*}
\text{scaled\_pt} &= ct_1 \cdot sk + ct_0 \mod q \\
\text{decrypted\_poly} &= \left\lfloor \frac{t \cdot \text{scaled\_pt}}{q} \right\rceil \mod t \\
\text{Decrypted Plaintext} &= \text{decrypted\_poly}
\end{align*}
\]

The RLWE homomorphic encryption scheme provides a powerful tool for
performing computations on encrypted data while preserving data privacy.
The scheme involves key generation, encryption, and various homomorphic
operations, making it suitable for secure computations in various
applications.
